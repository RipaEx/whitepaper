%----------------------------------------------------------------------------------------
%	CHAPTER 2: Introduction
%----------------------------------------------------------------------------------------

\chapterimage{chapter_head_2_London.jpg} % Chapter heading image

\chapter{Introduction}
The industry of virtual currencies has a high bar to entry from a technical point of view for the average user
and a high bar to entry from an economical point of view for the average entrepreneur for buying a reliable crypto 
exchange source code, hiring professional DevOps personnel, hiring customer support specialists, complying with national and 
international AML/KYC regulations and having liquidity from day one of the exchange start of operations. 
We want to lower this barrier because to entry this industry because \textbf{running an exchange is HARD} 
and we want you to focus on things that matter, not on the caveats that the industry requires because
you want to start a business in this industry and you need the source code to do it.

To addition to this point, starting an exchange requires a high level of investment form your venture capital reference
and the profit of your exchanges operations are not guaranteed in the first 5 years.

When building a professional exchange service we think that the source code of your exchange and the liquidity to offer to your clients
from day one should be given to you free of charge: no more paying \euro150,000.00 to a company just to have a platform that works and for
which you need to pay another \euro100,000.00 - 150,000.00 just to brand it and customize it for your needs so you can tie your 
business to a company that may go bankrupt in the future and found you in trouble as you never had the source code of the product
your business relies on.

\textbf{We believe that all of this should be free} and we should offer you the best technology in the market so you can focus on your business
while we focus on building the technology to run your business in an efficient, secure, responsive and productive way. That is why RipaEx 
is focusing on building a network of exchanges focusing on an exchange architecture that is \textit{efficient, secure, UI responsive, compliant and customizable}
so each exchange in the network can rely on solid foundations while customizing its single exchange instance for the needs of
the business entity and having its customization easily accessible to you.

For reaching that goal, we chosen to build our Ripa Liquidity Service Provider technology on top of ARK - a blockchain for consumer adoption - 
which primary focus is increasing consumer adoption for blockchain technologies focusing on two critical areas: \underline{A Fast Secure Core Technology}
and \underline{Practical Services for Real People}. ARK ecosystem is still at its early stage of development: in future implementations 
there will be the possibility to run smart contracts natively on the ARK 2.0 blockchain, this will permits this blockchain technology to compete 
with Ethereum from a technological point of view. RLSP will permits to share the same liquidity-orderbook to each exchange in the Ripa Network based on
single exchange installation privacy rules and preferences in the admin area of your individual exchange instance.

The Ripa Founder Team (RFT), as presented on ripaex.io, acts in the name of the Ripa Crew. The RFT is responsible for the proper use of 
funds collected under the Token Exchange Campaign (RIPA - TEC) presented below in this document.

The RFT undertakes that the result of this TEC will be used exclusively for the financing of the \emph{RipaEx} project as explained in this 
whitepaper - which will be made available on the collection platform: tec.ripaex.io - and which should result in the creation of a 
legal entity whose name will be \emph{RipaEx}. The creation of this company is scheduled for the first quarter of 2019.

To this end, RFT intervenes on behalf of \emph{RipaEx, a company in the process of being incorporated}.

\section{Key Terminology}
	\begin{description}
		\item[\textsc{Ripa Exchange:}] a FIAT $\Leftrightarrow$ CRYPTO exchange (a cryptocurrency exchange) based on the source code
		of Peatio \cite{peatio}
		\item[\textsc{Ripa Blockchain:}] a DPOS blockchain in which liquidity is shared for all the exchanges in the Ripa network
		\item[\textsc{Ripa Token - XPX:}] a cryptographically secure token exchanged on the Ripa blockchain based on the DPOS protocol
		\item[\textsc{RIPA:}] the DPOS financial ecosystem composed of Ripa Exchange and Ripa Blockchain
		\item[\textsc{RIPAEX:}] the name of the project, project website and hosted domain
		\item[\textsc{RLSP:}] Ripa Liquidity Service Provider, a shared orderbook to exchange orders and liquidity between exchanges in the same Ripa network
		\item[\textsc{RipaEx ICO:}] the name of fixed exchange rate token exchange period composed of both the phases of PreSale and RIPA TEC
		\item[\textsc{ARK:}] a platform for consumer adoption of blockchain technologies \cite{ark}
		\item[\textsc{ACES:}] Ark Contract Execution Services \cite{aces} provides simple protocols and tools for building a robust 
		blockchain service marketplace based on the ARK SmartBridge technology
		\item[\textsc{``,'' or ``.'':}] The Anglo-Saxon use of decimal points and commas to represent numbers has
		been chosen for the purposes of this document: that is to say that a “.” represents a decimal point, and a “,”
		distinguishes between multiples of thousands, millions and billions.
    \end{description}

%\pagebreak
\section{Roadmap}
There are essentially four phases to the RipaEx project:
\tcbset{roadmapBox/.style={colback=yellow!10!white,colframe=azure(colorwheel),
equal height group=nobefaf,width=(\linewidth-1pt), height from=4cm to 8cm, nobeforeafter,
	center title,
	valign=top, halign=left}}
\begin{center}
	\begin{tcolorbox}[roadmapBox,
		title=\textbf{\textsc{Funding the project: XPX PreSale and RIPA TEC (WP2)}}]

		This phase recognizes the existence of interest in this market development
		from across the World concerning the lowering of the entry level for building a cryptocurrency exchange.
		It aims to make the first comprehensive analysis of this state of the art to form the basis of the later project phases and
		build the first working prototype of a centralized exchange based on Peatio.\\
		\vspace{1cm}
		\centering\textbf{\textsc{Phase ending January 2019}}.
	\end{tcolorbox}
	\resizebox{0.05\textwidth}{26pt}{$\Downarrow$}
	\begin{tcolorbox}[roadmapBox,
		title=\textbf{\textsc{First exchange opening and development of tools and resources (WP3)}}]

		The second phase takes the results of the first 
		and develops from them a set of tools and resources which provide concise and comprehensible guidance to market actors in any
		Country. With the first instance of Ripa Exchange running we can make first contact with other economical players in the industry.\\
		\vspace{1cm}
		\centering\textbf{\textsc{Phase ending June 2019}}.
	\end{tcolorbox}
	\resizebox{0.05\textwidth}{26pt}{$\Downarrow$}
	\begin{tcolorbox}[roadmapBox,
		title=\textbf{\textsc{Dissemination (WP 7/8) and Project Coordination (WP1)}}]

		During the full duration of the project, 
		dissemination activities (WP 7/8) are carried out in which results from the individual work packages are disseminated 
		to relevant target groups including project partners, RipaEx investors, exchanges managers, banking partners as well 
		as relevant target groups. This phase covers a wide range of dissemination techniques, from printed and
		electronic handbooks to workshops and training sessions, hackatons, ongoing networks, all having the
		ultimate goal of defining a standard for exchange communication among public and private entities. 
		An overarching work package is concerned with the management of the project from start to finish, 
		ensuring proper coordination, quality assurance and budgetary control (WP1).
	\end{tcolorbox}
	\resizebox{0.05\textwidth}{26pt}{$\Downarrow$}
	\begin{tcolorbox}[roadmapBox,
		title=\textbf{\textsc{Development of hybrid-decentralized exchange (WP 4-6)}}]

		Using the tools and resources developed in WP3, 
		Work packages 4-6 focus on bringing collected knowledge and tools into practice. The three work packages reflect three major
		focal points (and target groups) within the network of exchanges created for establishing successful 
		demonstrations on a local scale: incorporations of local Ripa Exchanges (WP4), technical analysis for the 
		Ripa Liquidity Service Provider (WP5), and first MVP of the hybrid decentralized exchange (WP6). The demonstration
		phase forms the heart of the RipaEx action; WP 2 and 3 are focused on providing
		deliverables (e.g. tools) that enable successful and efficient demonstration activities.\\
		\vspace{1cm}
		\centering\textbf{\textsc{Phase ending January 2021}}.
	\end{tcolorbox}
\end{center}

\section{RipaEx Partners - RipaEx Governance}\index{RipaEx Partners - RipaEx Governance}
Most of the partners are entrepreneurs in the virtual currency industry, but a research institute 
and Financial Organizations are also represented. The partners are:
\begin{itemize}
	\item \textbf{Coordinator}: RipaEx SCIC 
	\item \textbf{CoBeneficiaries}: Ripa Exchange Ltd
\end{itemize}
\todo{TODO: Add RipaEx partners here}
\subsection{RipaEx Governance}\index{RipaEx Governance}
Creation of a new Ripa Exchange in the Ripa network, release of RCF funds for the creation of new exchanges will be
decided with a majority voting system among all the delegates of the Ripa network. The customization of the individual
exchange instance, otherwise, will be decided exclusively from the company that manages that single exchange
instance.

The RipaEx project will be incorporated as a collective interest company for profit as soon as the Ripa Blockchain
will be stable, instead the company Ripa Exchange Ltd will manage the first individual crypto asset marketplace
instance that will be opened to public in the first quarter of 2019.

\section{Project Summary}
\begin{enumerate}
	\item \textbf{\textsc{what}}: RipaEx is a project to facilitate the uptake of standards to share liquidity between crypto assets marketplaces. 
	The objective of RipaEx is the promotion of shared source code for wallets and exchanges in the virtual currency industry: 
	it is the aim of this reference document to give in-depth information to prospective exchange developers,
	or exchange managers, to enable correct decision-making and to ensure success for their proposed projects. 
	It seeks to analyze the real potential in the Country of application for a network of cryptocurrency exchanges, 
	and its place in the market.
	\item \textbf{\textsc{what}}: crypto assets are an alternative to centralized assets managed by (country-specific) stock exchanges. 
	Although certain stock exchanges gives the possibility to their users to verify and manage the assets they own, the verification process
	is not always transparent, that is the reason because from 2009 \cite{bitcoin} onwards a new types of (community-verifiable) assets 
	have been implemented to give small, medium and big investors complete transparency in the managing of their investments
	assets.
	\item \textbf{\textsc{when}}: recent developments at European Union level and worldwide are transforming both how virtual currencies
	are treated and the way ICO (Initial Coin Offering) are legislated. These combined developments 
	have made the use and production of virtual currencies an increasingly favorable prospect. \\
	In October 2015 the European Court of Justice ruled that bitcoin and other cryptocurrencies are exempt from VAT taxation. \\
	In July 2016 the European Commission adopted proposals for legislation to amend the 4th Anti-Money Laundering Directive (4AMLD) that
	will bring virtual currencies exchanged and wallet providers into the EU's anti-money laundering framework \cite{EUAMLCrypto}.\\
	In February 2018 the European Commission launched the EU Blockchain Observatory and Forum \cite{EUBOaF} to highlight key developments 
	in blockchain technology, promote European actors and reinforce European engagement with multiple stakeholders involved in blockchain activities.
	\item \textbf{\textsc{why}}: however there is still very little regulation performed on ICOs and only United States of America at the moment has undergone
	a legislation defining ICO tokens as securities and property. \cite{SECICO}
	\item \textbf{\textsc{who}}: the results indicate that medium tech savvy from 18 to 45 is the average user of
	virtual currencies although the corporate finance companies are also starting to put 
	virtual currencies schemes inside their portfolio especially since the presentation 
	of the Bitcoin futures contract from CME Group Inc. in the stock exchange of Chicago
	last 18th of December 2017.
	\item \textbf{\textsc{how much}}: total virtual currencies market capitalization has been estimated around 317 B USD\footnote{Coinmarketcap data April 2018}
	and is predicted to grow to 5,000 B USD in the next ten years span \cite{cryptoMCTenYears}.
	\item \textbf{\textsc{where}}: local authorities are working with National Governments to make sure local exchangers
	in the national territory are complying with national and international AML/KYC regulations.
	Venture capitals and Angel Investors are starting to release financing solutions to start-ups 
	in the Fintech industry all over the world from America to Asia passing through Europe and some 
	Countries are starting state-owned cryptocurrencies schemes to test the exchange of goods \& services
	on those (distributed ledger) technologies \cite{petro}.
	\item \textbf{\textsc{how much}}: the average cost for starting your own crypto asses marketplace is around \euro 150,000.00 only 
	for a running instance of your exchange platform:
	to that you need to add costs to customize the platform before launch and in the future, advertise your new business, running costs for servers,
	network operators, support center staff and legal department to comply with your State AML/KYC legislations and general company laws.\\
	\textbf{That is the reason because we think owning the source code of your exchange software is the best way to run a business in this industry}.
	\item \textbf{\textsc{how}}: the main problems encountered in opening a FIAT $\Leftrightarrow$ CRYPTO marketplace is to find trusted
	banking partners to comply with the many different AML/KYC rule and procedures to exchange virtual currencies
	to FIAT currencies.
	\item \textbf{\textsc{what}}: classical types of exchange operations are: 
		\begin{itemize}
			\item \textbf{one-way exchange}: in which a centralized application has all the liquidity to offer to its potential users
			\item \textbf{two-way exchange}: in which a centralized or decentralized platform match the selling requests with the buying requests
			of its users
		\end{itemize}
		On this, a sub-classification is also necessary:
		\begin{itemize}
			\item \textbf{FIAT $\Leftrightarrow$ CRYPTO exchange}: in which exchange operations are performed between 
			FIAT\footnote{Traditional central banks owned currencies like EUR, USD, GBP, JPY, others...} 
			currencies and virtual currencies
			\item \textbf{CRYPTO $\Leftrightarrow$ CRYPTO exchange}: in which exchange operations are performed only between virtual currencies
		\end{itemize}
	You can build a matrix based on the four configurations above to build the exchange operation platform of your needs.
	\begin{tcbraster}[raster columns=3,raster rows=1,raster height=0.8cm,
		valign=center, halign=center,
		enhanced,size=small,sharp corners,colframe=azure(colorwheel),coltext=white,
		colback=azure(colorwheel),fit algorithm=hybrid* ]
		\tcboxfit{}
		\tcboxfit{\textbf{one-way}}
		\tcboxfit{\textbf{two-way}}
	\end{tcbraster}
	\begin{tcbraster}[raster columns=3,raster rows=2,raster height=5cm,
		valign=center, halign=center,
		enhanced,size=small,sharp corners,colframe=silver,coltext=black,
		colback=silver,fit algorithm=hybrid* ]
		\tcboxfit{\textsc{\textbf{FIAT $\Leftrightarrow$ CRYPTO}}}
		\tcboxfit{\tcbfontsize{0.8} fast application to buy cryptocurrency}
		\tcboxfit{\tcbfontsize{0.8} exchange with banking partnership}
	
		\tcboxfit{\textsc{\textbf{CRYPTO $\Leftrightarrow$ CRYPTO}}}
		\tcboxfit{\tcbfontsize{0.8} fast application to buy cryptocurrency}
		\tcboxfit{\tcbfontsize{0.8} exchange without banking partnership}
	\end{tcbraster}	

	\item \textbf{\textsc{with what}}: the specifications to look when choosing for an exchange platform to run are:
		\begin{enumerate}[label*=\arabic*.]
			\item \textbf{Code}: Open Source, Closed Source or hybrid solution between the two
			\item \textbf{Modularization}: separation between exchange engine (orders matching engine), UI and user registry
			\item \textbf{UI responsiveness}: responsiveness of UI on all devices (desktop and mobile)
			\item \textbf{Compliance}: with current industry standards and rules \& regulations
			\item \textbf{Customization}: of the exchange engine, trading currencies, UI and other aspects of the crypto asset marketplace platform
			\item \textbf{Security} of the funds: saving in cold wallets and hot wallets configurable
			\item \textbf{Transparency} of the funds: proof of solvency of the exchange
			\item \textbf{Multi-Accounts trading}: easy to configure new virtual currency protocols
			\item \textbf{Multi-Accounts users}: possibility to interact with user accounts from Google, Facebook, Twitter 
			to login into the platform and FIDO Alliance security standards for personal credentials
		\end{enumerate}	
	\textbf{Those are not only technical decisions to be made but also economical} especially the owning of the source code of your crypto asset
	marketplace platform is fundamental to make future customization of your exchange in an independent way compared to rely on a single
	software house that makes the customizations for you at a high cost.
	\item \textbf{\textsc{how}}: options for finding users for your exchange operations are: targeted marketing campaigns, innovative features in the industry,
	fee level based on trading quantities, bonuses for first registration and trading quantities, affiliate marketing for paying users to bring their friends
	to your exchange.
	\item \textbf{\textsc{how}}: for setting-up a crypto asset marketplace a project must take into account the following legislation:
		\begin{itemize}
			\item \textbf{AML/KYC}: \textit{Fourth Anti-Money Laundering Directive} if business set up in the European Union \cite{4AMLD} or the AML/KYC reference implementation
			to your crypto asset marketplace Country of incorporation (as an example \textit{Intelligence Reform \& Terrorism Prevention Act of 2004}
			written by FinCEN in the United States of America).\\
			International recommendations for undergoing AML/CFT verifications are given by the Financial Action Task Force on Money Laundering \cite{FATF}.
			\item \textbf{Payment Licence}: by far the biggest and most arduous task with regards to legitimising the 
			FIAT $\Leftrightarrow$ CRYPTO exchange operations is obtaining a \textit{PSD Licence} \cite{PSD}. 
			The PSD license follows Council Directive 2007/64/EC and is applied in each country via its own national laws. 
			Costs of an PSD license can vary between hundred thousand euro to one order of magnitude greater depending on the exchange volume.
		\end{itemize}
	\item \textbf{\textsc{who}}: the nature of the business under consideration by the RipaEx project (small scale,
	localized FIAT $\Leftrightarrow$ CRYPTO exchanges operations), means that each enterprise likely to have 7 or 8 staff: N.2 developers, 
	N.1 network/security operator, N.1 administrative, N.2 client support operators, N.1 legal and tax advisor. \\
	The turnover of such an enterprise however, because of the high value of the end product, is likely to be more than 
	\euro 350,000 a year and could be several times higher. A business of
	this scale lends itself to the following possible company structures: A simple partnership;
	A limited company; A non-profit company or social enterprise; A worker co-operative.
	Financial Agencies are potential key actors, but the type of business they can set up will
	depend on their legal status which does vary from country to country.
	\item \textbf{\textsc{how}}: potential sources of funds for a small-medium sized crypto asset marketplaces are: Bank Loans; Low
	Interest Loan Schemes; Commercial Credit; Equity financing; Business Angels venture
	capital. Having a robust Business Plan and financial guarantees are essential elements
	for securing funding. The European Investment Fund (EIF) of the European Investment Bank (EIB), 
	offers support in the form of guarantees for SMEs.
	\item \textbf{\textsc{why}}: the arguments for crypto asset marketplaces are for financial freedom,
	decentralizing of the value-transferring operations, and real ownership of your money. 
	There are other benefits, well documented, such as faster payments, long term gain based on deflationary economy and prediction of 
	Great Depressions like the one that hit the global economy in 2008. But
	above all, virtual currencies are the only direct competitor to centralized value-transferring operations done by central banks.
	\item \textbf{\textsc{why}}: there is consensus in the literature that the use of virtual currencies in place of fiat currencies will result in 
	higher financial freedom especially as they fit into the Austrian school of economy \cite{austrianTheory} for lots of economic experts
	\cite{misesItalia}, although bitcoin is it better used as means of payment and is lacking the stability to be a healthy
	reserve of value.
	\item \textbf{\textsc{why}}: benefits of virtual currencies schemes like bitcoin are in a deflationary economy, in limited emission of money,
	in near instant and without border value transfer operations, in relative anonymity and in the separation between the entities 
	that produce the money (miners) from the entities that code the economic theory behind it (developers).
	\item \textbf{\textsc{how}}: securing assets on the blockchains means basically performing three operations
		\begin{enumerate}[label*=\arabic*.]
			\item \textbf{Generating a random private key}
			\item \textbf{Converting the private key generated in (1) into a public key}: a common protocol making this conversion
			in the virtual currencies industry is the ECDSA curve algorithm
			\item \textbf{Converting the public key generated in (2) into a virtual currency address}: common protocols for making this conversion
			are hash functions SHA-256, Base58 encoding, Base32 encoding
		\end{enumerate}
	At this point any value sent to the virtual currency address generated in (3) is secured on the blockchain of choice and accessible
	only from the owner of the relative private key generated in (1).
	\item \textbf{\textsc{where}}: the two critical factors affecting the cryptocurrency industry are competion from banks and 
	State legislation (bans, etc).
	Although a harmonization throughout Europe would be beneficial to development of the industry both in terms of 
	taxation and warranty approvals, this is currently not the case. Each country worldwide has its specific legislation and tax
	regime for all exchanges operations involving FIAT money, and State banning is going to completely liberalization 
	of this activities like European Union to complete banning and imprisonment of operators in this industry like Bangladesh
	\cite{bitcoinLegality}. 
	\item \textbf{\textsc{where}}: the Asiatic countries of South Korea, China and Japan are the leader in the field of cryptocurrencies for number of 
	transactions for over 9 years with a proactive approach and favorable tax regime. At the beginning of 2017 in Japan bitcoin
	has been declared legal tender but China has recently declared illegal token sale and exchanges and local cryptocurrencies
	marketplaces are closing down.
	\item \textbf{\textsc{where}}: any assessment of your local market should include: number of potential users to reach, 
	type of exchange to incorporate (FIAT $\Leftrightarrow$ CRYPTO or CRYPTO $\Leftrightarrow$ CRYPTO), type of virtual currencies protocol 
	to integrate (POW, DPOS, Masternodes, others...), types of services to offer (exchange only, advanced trading tools,
	payment processor, others...), if FIAT $\Leftrightarrow$ CRYPTO exchange number of FIAT payments processors to accept (PayPal, OKPay,
	MoneyPolo, others...), number of others exchanges in your region.
	\item \textbf{\textsc{who}}: there are a number of options for dealing with Warranty/Customer protection issues: 
	creating consumer pressure by making clear to the end users that the possession of the private keys of their
	virtual currency addresses make \textbf{liable} for any loss of the private keys meaning nobody can help
	them recovering their funds if the their private keys are lost. Creating consumer pressure to not leave funds
	on exchanges (``\textit{Be Your own Bank!!}''), making them choose the licensed exchanges in the market. 
	\item \textbf{\textsc{who}}: while it is very expensive to insure money exchanges operations and money transmitting operations, 
	examples of customer protections in the industry are: Kraken platform which is offering Mt. Gox users partial refund
	of their lost, BitFinex exchange has launched the BFX token to repay users for the lost of 120,000 bitcoin 
	(valued \$72,000,000 at the time of the hacking) caused by the change in the wallet configuration done by the exchange
	itself, Ethereum did a network fork on the block height 1,920,000 to stop the lost of ETH funds from The DAO, Decentralized
	Autonomous Organization that had a bug that permits users to claim their ETH back using recurrency in The DAO codebase,
	other hacking cases...
	\item \textbf{\textsc{Recommendations for all/law compliance}}: if you intend to incorporate a FIAT $\Leftrightarrow$ CRYPTO 
	exchange you should
	focus from the first instance on law compliance by studying the AML/KYC laws of the country of incorporation and
	finding bank partners to work with. Local financial Authority can help to comply with rules \& regulations and 
	local cryptocurrencies foundations can help you to tune your exchanges operations to perform targeted
	operations based on the customers interests in the country of incorporation promoting cryptocurrency-friendly 
	users in the area of interest.
\end{enumerate}
