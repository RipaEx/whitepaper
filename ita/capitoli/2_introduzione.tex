%----------------------------------------------------------------------------------------
%	CHAPTER 2: Introduction
%----------------------------------------------------------------------------------------

\chapterimage{chapter_head_2_London.jpg} % Chapter heading image
\chapter{Introduzione}
Il livello di accesso all'industria delle valute virtuali è piuttosto elevato da un punto di vista tecnico per l'utente medio, 
ed ha un livello di accesso economicamente elevato per l'imprenditore che vuole avviare un'attività in questo campo, valore dato 
dall'acquisto del codice sorgente dell'exchange di criptovalute, dal reclutamento di personale che gestirà l'infrastruttura informatica,
per reclutare operatori di supporto tecnico, per sottostare alle leggi nazionali ed internazionali in materia di AML/KYC, per avere liquidità
dal primo giorno di avvio delle operazioni di cambio. Con questo progetto, Ripa Exchange, vuole abbassare questo livello di ingresso 
perchè \textbf{gestire un exchange è DIFFICILE} ed il Team RipaEx vuole che il lettore e futuro imprenditore nel campo delle valute
virtuali si concentri su cose veramente importanti non cavilli burocratici che l'industria richiede perchè TU vuoi avviare un'attività
in questo campo e tu necessiti del codice sorgente per avviarla, codice sorgente che deve essere concesso gratuitamente alla tua attività.

Per rafforzare questo principio economico il punto è che per avviare un exchange è richiesto un investimento esiguo dal tuo 
venture capital di riferimento e anche con un tale investimento il ritorno dei fondi investiti all'angel investor non sono garantiti
nei primi 5 anni di attività.

Per avviare un servizio professionale di cambiavaluta virtuale crediamo che il codice sorgente dell'exchange e la liquidità da offrire ai tuoi clienti
dal primo giorno di attività dovrebbero esserti forniti gratuitamente: non deve essere accettabile pagare \euro150.000,00 ad una software house
solo per avere una piattaforma che funzioni e per la quale dovete erogare altri \euro100.000,00-150.000,00 per brandizzarla, personalizzarla
per i vostri scopi legandovi in questo modo ad una singola software house che potrebbe fallire in futuro e che vi farebbe ritrovare
senza poter fare ulteriori modifiche al vostro exchange in quanto non vi hanno mai consegnato il codice sorgente della piattaforma
su cui il vostro business fa affidamento.

\textbf{Noi crediamo che tutto questo debba essere gratuito}: dobbiamo offrirti la miglior tecnologia nel mercato attuale in modo tale che tu 
possa focalizzarti sulla tua attività mentre noi ci focalizziamo in sviluppare la tecnologia per gestire la tua attività in maniera efficiente, 
sicura, responsiva e produttiva. Questo è il motivo per cui RipaEx si prefigge lo scopo di sviluppare una rete di exchange
focalizzandosi su di un'architettura che è \textit{efficiente, sicura, responsiva, compliant e personalizzabile} in modo tale che ogni exchange
nella rete può basarsi su di fondamenta solide mentre è comunque possibile modificare la singola istanza secondo le personalizzazioni 
che la compagnia che gestisce quell'istanza richiede.

Per raggiungere questo scopo abbiamo scelto di sviluppare la tecnologia Ripa Liquidity Service Provider basata su ARK - a blockchain for consumer adoption - 
il cui obbiettivo primario è incrementare l'adozione delle tecnologie blockchain agli utenti finali focalizzandosi su due aree critiche: 
\underline{Una Efficiente e Sicura Tecnologia di Base} e \underline{Servizi Reali per Gente Comune}. L'ecosistema ARK è ancora ad uno stadio
iniziale di sviluppo: la futura implementazione di ARK 2.0 permetterà di eseguire smart contract nativi sulla blockchain, che permetteranno
alla blockchain ARK di competere con la sua concorrente Ethereum da un punto di vista tecnologico. RLSP permetterà a tutti gli exchange della rete Ripa
di scambiare liquidità condividendo lo stesso libro degli ordini in base ai parametri di condivisione della singola istanza dell'exchange Ripa
configurabili nella tua area amministratore.

Il Ripa Founder Team (RFT), come presentato in ripaex.io, opera nel nome del gruppo Ripa. L'RFT è responsaile per l'uso corretto dei finanziamenti
ottenuti nella campagna di scambio token RIPA TEC come spiegato di seguito in questo documento.

Il RFT dispone che i fondi delle fasi di PreSale e RIPA TEC saranno usati esclusivamente per il finanziamento del progetto 
\emph{RipaEx} come spiegato in questo whitepaper, che saranno resi disponibili per estrazione nella piattaforma apposita resta disponibile
più tardi quest'anno tec.ripaex.io, e che dovranno risultare nella creazione di una entità legale che sarà chiamata \emph{RipaEx}. La creazione di tale 
compagnia è pianificata per il primo quadrimestre del 2019.

Fino a tale data RFT opera invece di \emph{RipaEx, una compagnia nel processo di essere incorporata}.

\section{Terminologia}\index{Terminologia}
\begin{description}
	\item[\textsc{Ripa Exchange:}] un exchange FIAT $\Leftrightarrow$ CRYPTO sviluppato partendo dal codice sorgente di Peatio \cite{peatio}
	\item[\textsc{Ripa Blockchain:}] una blockchain DPOS in cui la liquidità ed il libro mastro degli ordini sono condivisi tra tutti gli exchange della rete
	\item[\textsc{Ripa Token - XPX:}] un token crittograficamente sicuro tramite il protocollo DPOS scambiato sulla blockchain Ripa 
	\item[\textsc{RIPA:}] l'ecosistema finanziario DPOS composto da Ripa Exchange e Ripa Blockchain
	\item[\textsc{RIPAEX:}] il nome del progetto, sito web del progetto e domini del progetto
	\item[\textsc{RLSP:}] Ripa Liquidity Service Provider, un libro ordini condiviso tra gli exchange nella stessa rete Ripa
	\item[\textsc{RipaEx ICO:}] il nome del periodo di cambiavaluta a tasso fisso composto dalle fasi di PreSale e RIPA TEC	
	\item[\textsc{ARK:}] una piattaforma per l'adozione consumer di tecnologie blockchain \cite{ark}
	\item[\textsc{ACES:}] Ark Contract Execution Services \cite{aces} fornisce semplici protocolli e strumenti per costruire robusti
	mercati di scambio tra blockchain basati sulla tecnologia ARK SmartBridge
	\item[\textsc{``,'' o ``.'':}] la notazione italiana per virgola decimale e punti di migliaia è stata scelta per la stesura di questo 
	whitepaper: detto questo “,” rappresenta un punto decimale e “.” rappresenta la separazione tra migliaia e multipli di migliaia 
\end{description}

%\pagebreak
\section{Roadmap}
Saranno percorse essenzialmente quattro fasi del progetto RipaEx:
\tcbset{roadmapBox/.style={colback=yellow!10!white,colframe=azure(colorwheel),
equal height group=nobefaf,width=(\linewidth-1pt), height from=4cm to 8cm, nobeforeafter,
	center title,
	valign=top, halign=left}}
\begin{center}
\begin{tcolorbox}[roadmapBox,
	title=\textbf{\textsc{Finanziamneto del progetto: XPX presale e RIPA TEC (WP2)}}]

	Questa fase riconosce l'esistenza di un interesse in questo sviluppo di mercato
	da un capo del Mondo all'altro in riguardo l'abbassamento del livello economico di ingresso
	per sviluppare exchange di criptovalute.
	L'obbiettivo è di eseguire la prima analisi omnicomprensiva dello stato dell'arte per fondare
	le basi delle fasi sucessive del progetto e per sviluppare il primo prototipo
	funzionante di un exchange centralizzato con il codice di Peatio.\\
	\vspace{1cm}
	\centering\textbf{\textsc{Prevista fine della fase: Gennaio 2019}}.
\end{tcolorbox}
\resizebox{0.05\textwidth}{26pt}{$\Downarrow$}
\begin{tcolorbox}[roadmapBox,
	title=\textbf{\textsc{Apertura del primo exchange e sviluppo di strumenti e risorse (WP3)}}]

	La seconda fase prende i risultati della prima e sviluppa da qui
	una serie di strumenti e risorse che offrono guida concisa ed omnicomprensiva per 
	i player senza considerare lo Stato di incorporazione/sviluppo del progetto. Con la prima
	istanza di Ripa Exchange in esecuzione/aperta al pubblico i primi contatti con altri player economici
	nell'industria delle valute virtuali possono essere presi.\\
	\vspace{1cm}
	\centering\textbf{\textsc{Prevista fine della fase: Giugno 2019}}.
\end{tcolorbox}
\resizebox{0.05\textwidth}{26pt}{$\Downarrow$}
\begin{tcolorbox}[roadmapBox,
	title=\textbf{\textsc{Esecuzione, diffusione (WP 7/8) e Coordinamento del Progetto (WP1)}}]

	Durante tutta la durata del progetto,
	attività di diffusione (WP 7/8) sono eseguite nelle quali i risultati dei singoli
	pacchetti di lavoro individuali sono consegnati ai relativi gruppi obbiettivo inclusi
	partners del progetto, investitori RipaEx, manager di exchange, partner bancari come anche
	tutti gli altri gruppi obbiettivo rilevanti al progetto RipaEx in esecuzione.
	Questa fase copre una vasta scala di tecniche di diffusione, da libri stampati ed 
	elettronici, a workshop e sessioni di training, hackatons, tutti aventi 
	come obbiettivo di definire standard per la comunicazione tra exchange gestiti da entità 
	pubbliche e private nell'industria delle valute virtuali.
	Un pacchetto di lavoro globale riguardante la gestione del progetto dall'inizio alla fine, 
	assicurando coordinazione adeguata, quality assurance e controllo del budget (WP1).
\end{tcolorbox}
\resizebox{0.05\textwidth}{26pt}{$\Downarrow$}
\begin{tcolorbox}[roadmapBox,
	title=\textbf{\textsc{Sviluppo di un exchange ibrido-decentralizzato (WP 4-6)}}]

	Usando gli strumenti e le risorse sviluppate in WP3, il pacchetto di lavoro 4-6
	si focalizza di mettere in pratica la conoscenza acquisita e gli strumenti. I tre
	pacchetti di lavoro riflettono tre punti focali (e gruppi target) nel network di 
	exchange creato per installare demo di successo su scala locale, nazionale ed internazionale:
	l'incorporazione di istanze di Ripa Exchange locali (WP4), analisi tecnica della tecnologia
	Ripa Liquidity Service Provider (WP5), e primo MVP dell'exchange ibrido decentralizzato (WP6).
	La fase demo dal cuore dell'azione RipaEx; WP 2 e 3 focalizzano in produrre consegne (ad esempio
	strumenti) che abilitano attività di produzione demo efficienti e di successo.\\
	\vspace{1cm}
	\centering\textbf{\textsc{Prevista fine della fase: Gennaio 2021}}.
\end{tcolorbox}
\end{center}

\section{RipaEx Partners - RipaEx Governance}\index{RipaEx Partners - RipaEx Governance}
La maggior parte dei parnter sono imprenditori nell'industria delle valute virtuali, tuttavia
un Istituto di Ricerca ed Organizzazioni Finanziari sono anche presenti. I partner sono:
\begin{itemize}
	\item \textbf{Coordinatore}: RipaEx SCIC 
	\item \textbf{CoBeneficiari}: Ripa Exchange Ltd \todo{TODO: AGGIUNGERE PARTNER RIPAEX}
\end{itemize}

\subsection{RipaEx Governance}\index{RipaEx Governance}
La creazione di nuovi exchange nella rete Ripa, il rilascio dei fondi dell'RCF per la creazione di nuovi exchange saranno decise
su votazione maggioritaria dei delegati della rete Ripa. Le modifiche del singolo exchange, invece, saranno appannaggio 
esclusivo della compagia che gestisce la singola istanza dell'exchange.

Il progetto RipaEx si incorporerà come società di interesse collettivo per profitto appena la rete Ripa sarà stabile, mentre Ripa Exchange Ltd
società a gestione della prima istanza di exchange in esecuzione sarà incorporata come compagnia privata nel primo quarto
del 2019.

\section{Sommario del Progetto}
\begin{enumerate}
	\item \textbf{\textsc{cosa}}: RipaEx è un progetto per facilitare la presa di standard per condividere liquidità tra mercati di asset
	criptosicuri (crypto asset marketplace).
	L'obbiettio di RipaEx è la promozione di codice sorgente condiviso per portafogli ed exchange nell'industria delle valute
	virtuali: è obbiettivo di questo documento di riferimento di dare informazioni dettagliate per futuri
	sviluppatori di exchange oppure futuri imprenditori in questo campo, per abilitare processi di decision-making corretti e per 
	assicurare il successo dei loro progetti proposti.
	Il progetto RipaEx si prefigge l'obbiettivo di analizzare il potenziale reale di applicazione nel Paese scelto 
	di una rete di exchange e la sua posizione nel mercato.
	\item \textbf{\textsc{cosa}}: gli asset crittograficamente sicuri (aka: ``crypto assets'') sono un'alternativa agli asset emessi dalle borse
	valori dei vari Paesi che ne possiedono una. Anche se, tuttavia, molte borse valori mondiali danno la possibilità ai loro 
	utenti di verificare e gestire gli asset in loro possesso il processo non è sempre trasparente e/o semplice e questo è il motivo 
	per cui a partire dal 2009 \cite{bitcoin} un nuovo tipo di asset (verificabile dalla comunità) è stato implementato
	per dare a piccoli, medi e grandi investitori completa trasparenza nella gestione dei loro asset di investimento.
	\item \textbf{\textsc{quanto}}: recenti sviluppo a livello dell'Unione Europea e globalmente stanno trasformando sia come le valute virtuali
	sono trattate sia il modo in cui le ICO (Initial Coin Offering) sono legislate. Questi sviluppi combinati hanno messo
	l'uso e la produzione di valute virtuali in una luce favorevole in costante ascesa. \\
	In Ottobre 2015 la Corte Europea di Giustiza ha sentenziato che bitcoin ed altre valute virtuali sono esenti dalla tassazione IVA. \\
	In Luglio 2016 la Commissione Europea ha adottato proposal per ammendare la legislazione vigente 4th Anti-Money Laundering Directive (4AMLD) 
	che introduca le figure degli exchange di valute virtuali ed offritori di wallet all'interno del framework europeo di adeguata verifica
	dell'utente ed antiriciclaggio \cite{EUAMLCrypto}.\\
	In Febbraio 2018 la Commissione Europea lancia l'Osservatorio EU sulle Blockchain (EU Blockchain Observatory and Forum) \cite{EUBOaF} 
	per osservare gli sviluppi chiave nella tecnologia blockchain, promuovere attori europei e rafforzare l'ingaggio europeo con
	diversi stakeholders coinvolti in attività correlate alla blockchain.
	\item \textbf{\textsc{perchè}}: c'è ancora poca regolamentazione sulle ICO e solo gli Stati Uniti d'America al momento
	hanno scritto una legislazione definendo i token delle ICO come securities \cite{SECICO}.
	\item \textbf{\textsc{chi}}: i risultati delle indagini di mercato indicano che il medio tech savvy dai 18 ai 45 anni di età è
	l'utente medio delle valute virtuali tuttavia compagnie in finanza stanno anche comincianto ad inserire schemi di 
	valute virtuali all'interno dei loro portafogli specialmente dalla presentazione del contratto bitcoin futures 
	da parte del Gruppo CME Group Inc. nella borsa valori di Chicago lo scorso 18 Dicembre 2017.
	\item \textbf{\textsc{quanto}}: la capitalizzazione di mercato totale dell'industria delle valute virtuali è stata stimata attorno
	ai 317 miliardi di USD \footnote{Dati Coinmarketcap Aprile 2018}
	ed è prevista la crecita a 5.000 miliardi di USD nel prossimo timeframe di 10 anni \cite{cryptoMCTenYears}.
	\item \textbf{\textsc{dove}}: autorità locali stanno lavorando con i Governi Nazionali per essere certi che gli exchangers sul 
	territorio nazionale di riferimento seguono le procedure nazionali ed internazionali di AML/KYC.
	Venture Capital ed Angel Investors stanno cominciando a rilasciare soluzioni di finanziamento a start-up nell'industria delle
	valute virtuali - FinTech in tutto il mondo, dall'America all'Asia passando per l'Europa ed alcuni Paesi
	stanno avviando schemi di valuta virtuale statali per testare lo scambio di beni e servizi sulle tecnologie a libro 
	mastro condiviso (DLTs - Distributed Ledger Technologies) \cite{petro}.
	\item \textbf{\textsc{quanto costa}}: il costo medio per avviare il proprio crypto assets marketplace è attorno a \euro 150.000,00, 
	al pagamento di tale somma avrete unicamente un'istanza in esecuzione della vostra piattaforma di cambio: al suddetto prezzo dovete aggiungere i costi
	di personalizzazione della piattaforma prima del lancio ed in futuro, pubblicizzazione della piattaforma, costo di gestione dei server, 
	amministratori di rete, operatori del centro di supporto e costo del dipartimento legale per seguire le leggi AML/KYC dello Stato
	in cui avete deciso di incorporare il vostro exchange e per gestire l'amministrazione della compagnia dietro di esso.\\
	\textbf{Questi costi nascosti sono la ragione per cui noi di RipaEx riteniamo che possedere il codice sorgente del proprio exchange è il miglior
	modo per avviare e mantenere un'attività in questa industria}.
	\item \textbf{\textsc{come}}: il problema principale nell'eseguire un'operazione di cambio FIAT $\Leftrightarrow$ CRYPTO è di trovare partner bancari affidabili
	e nel seguire le differenti procedure AML/KYC del Paese o dei paesi di incorporazione.
	\item \textbf{\textsc{cosa}}: le operazioni di cambio valuta virtuale classiche sono le seguenti:
		\begin{itemize}
			\item \textbf{scambio ad una-via}: nella quale un'applicazione centralizzata ha tutta la liqudità da offrire ai potenziali utenti
			\item \textbf{scambio a due-vie}: nella quale un exchange centralizzato o decentralizzato accoppia richiesta di vendita con richiesta di aquisto degli utenti
		\end{itemize}
		Dalla classificazione data sopra una ulteriore sotto-classificazione è d'obbligo:
		\begin{itemize}
			\item \textbf{scambio FIAT $\Leftrightarrow$ CRYPTO}: nella quale lo scambio avviene tra valuta FIAT\footnote{Valute emesse dalle banche centrali: EUR, USD, GBP, JPY, altre...}
			e valuta virtuale
			\item \textbf{scambio CRYPTO $\Leftrightarrow$ CRYPTO}: nella quale lo scambio avviene tra valuta virtuale e valuta virtuale
		\end{itemize}
	Dalle quattro classificazioni sopra si può costruire una matrice con quattro possibili configurazioni per sviluppare la piattaforma di cambia valuta virtuale
	che preferite.
	\begin{tcbraster}[raster columns=3,raster rows=1,raster height=0.8cm,
		valign=center, halign=center,
		enhanced,size=small,sharp corners,colframe=azure(colorwheel),coltext=white,
		colback=azure(colorwheel),fit algorithm=hybrid* ]
		\tcboxfit{}
		\tcboxfit{\textbf{una-via}}
		\tcboxfit{\textbf{due-vie}}
	\end{tcbraster}
	\begin{tcbraster}[raster columns=3,raster rows=2,raster height=5cm,
		valign=center, halign=center,
		enhanced,size=small,sharp corners,colframe=silver,coltext=black,
		colback=silver,fit algorithm=hybrid* ]
		\tcboxfit{\textsc{\textbf{FIAT $\Leftrightarrow$ CRYPTO}}}
		\tcboxfit{\tcbfontsize{0.8} applicazione di acquisto veloce}
		\tcboxfit{\tcbfontsize{0.8} exchange con partnership bancaria}
	
		\tcboxfit{\textsc{\textbf{CRYPTO $\Leftrightarrow$ CRYPTO}}}
		\tcboxfit{\tcbfontsize{0.8} applicazione di scambio veloce}
		\tcboxfit{\tcbfontsize{0.8} exchange senza partnership bancaria}
	\end{tcbraster}	

	\item \textbf{\textsc{con cosa}}: le specifiche da guardare quando scegliere la propria piattaforma di exchange sono le seguenti: 
		\begin{enumerate}[label*=\arabic*.]
			\item \textbf{Codice}: Open Source, Closed Source oppure soluzione ibrida
			\item \textbf{Modularità}: separazione tra l'engine di cambio (engine di accoppiamento ordini), la UI e l'anagrafica utenti
			\item \textbf{Responsività della UI}: la UI deve essere responsiva su tutti i device (desktop e mobile)
			\item \textbf{Conformità}: con gli standard industriali correnti e con la legislazione vigente
			\item \textbf{Personalizzazione}: dell'engine di cambio, delle valute da offrire, della UI e di altri aspetti della piattaforma di cambio
			\item \textbf{Sicurezza} dei fondi nell'exchange: possibilità di configurare il salvataggio dei fondi in cold wallet e hot wallet
			\item \textbf{Trasparenza} dei fondi: proof of solvency dell'exchange
			\item \textbf{Multi-Accounts trading}: possibilità di configurare facilmente nuovi mercati da offrire agli utenti
			\item \textbf{Multi-Accounts users}: possibilià di loggarsi nella piattaforma tramite Google, Facebook, Twitter, e conformità
			con gli standard della FIDO Alliance per le credenziali personali
		\end{enumerate}	
	\textbf{Queste non sono solo decisioni tecniche da prendere ma anche economiche} specialmente il possesso o meno del codice sorgente della propria 
	piattaforma	crypto asset marketplace per permettere le future personalizzazioni del proprio exchange in maniera indipendente invece di affidarsi
	ad una singola software house che esegua le personalizzazioni per voi.
	\item \textbf{\textsc{come}}: possibili operazioni per trovare nuovi utenti per le vostre operazioni di cambia valuta virtuale sono:
	campagne di marketing ``ad hoc'', funzionalità innovative nell'industria, livelli di tassazione del trading basati sulla quantità, 
	bonus di registrazione e per quantità di scambiato giornaliero/settimanale/mensile, marketing affiliato per gli utenti che portano
	amici a scambiare nell'exchange.
	\item \textbf{\textsc{come}}: per incorporare una crypto asset marketplace un progetto deve tenere in considerazione la seguente legislazione:
		\begin{itemize}
			\item \textbf{AML/KYC}: \textit{Fourth Anti-Money Laundering Directive} se l'attività viene avviata nell'Unione Europea \cite{4AMLD} 
			oppure la legge AML/KYC di riferimento nel Paese di incorporazione (ad esempio \textit{Intelligence Reform \& Terrorism Prevention Act of 2004}
			emessa dalla FinCEN per incorporazione negli Stati Uniti d'America).\\
			Raccomandazioni internazionali per eseguire verifiche AML/CFT sono data dalla Financial Action Task Force on Money Laundering \cite{FATF}.
			\item \textbf{Licenza di Pagamento}: l'attività decisamente più ardua in riguardo alla legittimazione delle operazioni di cambio FIAT $\Leftrightarrow$ CRYPTO
			è l'ottenimento di una \textit{Licenza PSD} \cite{PSD}.
			La licenza PSD seguen la Direttiva del Concilio Europeo 2007/64/EC ed è applicata in ogni Paese attraverso le sue leggi nazionali.
			Il costo di una licenza PSD può variare da centinaia di migliaia di euro ad un ordine di grandezza superiore al seconda del volume di affari.
		\end{itemize}
	\item \textbf{\textsc{chi}}: la natura dell'attività in considerazione del progetto RipaEx 
	(una rete di piccoli exchange FIAT $\Leftrightarrow$ CRYPTO)
	implica che ogni exchange debba avere uno staff di 7-8 persone: N.2 sviluppatori, N.1 operatore di rete/sicurezza, 
	N.1 commerciale amministrativo, N.2 operatori di supporto tecnico, N.1 legale e consigliere sulla tassazione. \\
	Il turnover di una tale compagnia tuttavia, considerato l'alto valore aggiunto del prodotto scambiato, e possibile che raggiunga cifre
	superiori ad \euro 350.000 annui e possibilmente di un'ordine di grandezza superiore. Un'attività di questa dimensione
	suggerisce automaticamente le possibili forme societarie: una ditta individuale, una società a responsabilità limitata,
	una società non-profit o attività sociale, una cooperativa gestita dagli amministratori delegati dei vari exchange.
	Le Agenzie Finanziarie sono attori fondamentali nell'indicare questa scelta, tuttavia la tipologia di attività da
	incorporare dipende dallo status legale che si vuole dare alla stessa e questo varia da Paese a Paese.		
	\item \textbf{\textsc{come}}: una risorsa di fondi per una marketplace di dimensioni piccole-medie può essere: prestito bancario,
	prestito privato a basso interesse, credito commerciale, equity finanziario, venture capital di business angels. Avendo un
	business plan consolidato e garanzie finanziarie adeguate sono elementi chiave nell'assicurarsi i finanziamenti necessari
	allo sviluppo che si vuole eseguire. Il Fondo Europeo di Investimento (EIF) della Banca di Investimento Europea (EIB)
	offre supporto nella forma di garanzia per le SME (piccole e medie imprese).
	\item \textbf{\textsc{chi}}: gli argomenti per l'incorporazione di un crypto asset marketplace sono la libertà finanziaria, la
	decentralizzazione delle operazioni di trasferimento del valore ed il reale possesso del proprio denaro.
	Ci sono altri benefici, ben documentati, come pagamenti veloci, guadagno a lungo termine dato da un'economia deflazionaria, previsione
	delle depressioni economiche come quella del 2008, ma soprattutto le valute virtuali sono l'unico competitor diretto alle operazioni
	di trasferimento del valore centralizzate eseguite dalle banche centrali.
	\item \textbf{\textsc{perchè}}: c'è consenso in letteratura che l'uso delle valute virtuali invece delle valute fiat permette un livello di
	libertà finanziaria maggiore specialmente perchè queste tecnologie di trasferimento valore rientrano nella scuola
	di economia Austriaca \cite{austrianTheory} secondo molte fonti \cite{misesItalia} anche se tuttavia bitcoin dà il meglio come
	mezzo di scambio e mancherebbe della stabilità di tasso di cambio che la renderebbero una riserva di valore accettabile.
	\item \textbf{\textsc{perchè}}: benefici delle tecnologie di valuta virtuale come bitcoin sono in una economia deflazionistica,
	in un'emissione limitata di valuta, in trasferimenti di valore pressochè istantanei e senza barriere statali, in 
	relativo anonimato e nella separazione tra entità che producono la moneta (miner) ed entità che definiscono la teoria monetaria
	alla base della stessa (sviluppatori).
	\item \textbf{\textsc{come}}: securizzare asset sulla blockchain significa semplicisticamente eseguire le tre operazioni
		\begin{enumerate}[label*=\arabic*.]
			\item \textbf{Generazione di una chiave privata casuale}
			\item \textbf{Conversione della chiave privata generata in (1) in una chiave pubblica}: un protocollo comune per la conversione
			in questa industria è l'algoritmo ECDSA
			\item \textbf{Conversione della chiave pubblica generata in (2) in un indirizzo di valuta}: protocolli comuni per eseguire questa
			conversione sono le funzioni hash SHA-256, Base58 encoding, Base32 encoding
		\end{enumerate}
	A questo punto qualsiasi valore inviato all'indirizzo generato in (3) è garantito dalla doppia spesa nella blockchain scelta ed accessibile
	solo dal possessore della chiave privata generata in (1).
	\item \textbf{\textsc{dove}}: due fattori critici che interessando l'industria delle valute virtuali sono la concorrenza bancaria ed il 
	banning a livello statale. Anche se l'armonizzazione delle leggi europee in materia sarà di beneficio all'industria sia in termini di 
	tassazione sia in termini di garanzia per l'utilizzatore finale, al momento questo ancora non avviene. Ogni Paese su scala mondiale 
	ha la propria legislazione in materia ed i propri regimi di tassazione per tutte le operazioni di cambiavaluta virtuale,
	il banning statale invece va da livelli	di permissività totale come nell'Unione Europea al banning totale con 
	imprigionamento come in Bangladesh \cite{bitcoinLegality}. 
	\item \textbf{\textsc{dove}}: i paesi asiatici della Corea del Sud, Cina e Giappone sono sempre stati leader nell'industria delle valute virtuali
	negli ultimi 9 anni con un approccio proattivo in materia di tassazione. All'inizio del 2017 in Giappone il bitcoin è stato 
	dichiarato legal tender tuttavia in Cina recentemente sono state dichiarate illegali tutte le operazioni di cambiavaluta virtuale ed ICO 
	e gli exchange nel Paese asiatico stanno cessando le loro attività di cambio. 
	\item \textbf{\textsc{dove}}: qualsiasi valutazione del mercato locale dovrebbe comprendere: numero di potenziali utenti, tipo di exchange da 
	incorporare (FIAT $\Leftrightarrow$ CRYPTO oppure CRYPTO $\Leftrightarrow$ CRYPTO), tipi di protocolli da integrare (POW, DPOS, Masternode, altri...), 
	tipi di servizi da offrire (trading, trading avanzato, processore di pagamento, altri...), se l'exchange incorporato è di tipologia
	FIAT $\Leftrightarrow$ CRYPTO numero e tipologia di processori di pagamento accettati (PayPal, OKPAy, MoneyPolo, altri...), numero di exchange
	simili nella regione di incorporazione. 
	\item \textbf{\textsc{chi}}: ci sono numerose opzioni per offrire una polizza assicurativa sui fondi degli utilizzatori finali (Customer Protection):
	creare comprensione tra gli utenti che il possesso delle chiavi private dei fondi mette loro in \textbf{posizione di responsabilità} 
	nei confronti di un'eventuale perdita dei fondi, in quanto nessuno può recuperare le chiavi private una volta che sono perse. 
	Creare comprensione negli utenti per non lasciare i fondi negli exchange ("\textit{Be Your own Bank!!}"), lasciandogli scegliere a quale
	exchange del mercato affidarsi.
	\item \textbf{\textsc{chi}}: mentre è decisamente costoso assicurare la valuta e le operazioni di trasmissione del valore, esempi di customer protection
	in questa industria sono: la piattaforma Kraken ha recentemente permesso agli utenti Mt. Gox di richiedere rimborso tramite
	la loro piattaforma, l'exchange BitFinex ha emesso token BFX per ripagare i propri utenti della perdita di 120.000 bitcoin (pari a circa \$72.000.000
	all'epoca dell'hacking) dovuti al cambio del livello di sicurezza dell'exchange stesso, Ethereum ha eseguito uno fork al blocco 1920000
	per ovviare ad un bug nel codice della Decentralized Autonomous Organization che ha permesso la sottrazione di \$50.000.000 di ETH dallo smart contract
	di The DAO, altri casi di hacking...
	\item \textbf{\textsc{Raccomandazioni finali/conformità legislativa}}: se intendete incorporare un exchange 
	FIAT $\Leftrightarrow$ CRYPTO dovete focalizzarvi in prima istanza nella conformità legislativa in materia 
	di AML/KYC nel Paese di incorporazione, le fondazioni
	bitcoin locali possono aiutarvi in focalizzare le vostre operazioni di cambio basate sugli interessi degli utilizzatori finali 
	nel Paese di incorporazione, promuovendo ambienti \textit{cryptocurrency friendly} nell'area di interesse.
\end{enumerate}
